% ****** Start of file apssamp.tex ******
%
%   This file is part of the APS files in the REVTeX 4 distribution.
%   Version 4.0 of REVTeX, August 2001
%
%   Copyright (c) 2001 The American Physical Society.
%
%   See the REVTeX 4 README file for restrictions and more information.
%
% TeX'ing this file requires that you have AMS-LaTeX 2.0 installed
% as well as the rest of the prerequisites for REVTeX 4.0
%
% See the REVTeX 4 README file
% It also requires running BibTeX. The commands are as follows:
%
%  1)  latex apssamp.tex
%  2)  bibtex apssamp
%  3)  latex apssamp.tex
%  4)  latex apssamp.tex
%
\documentclass[prb,aps,twocolumn,preprintnumbers,amsmath,amssymb]{revtex4}
%\documentclass[preprint,showpacs,preprintnumbers,amsmath,amssymb]{revtex4}

% Some other (several out of many) possibilities
%\documentclass[preprint,aps]{revtex4}
%\documentclass[preprint,aps,draft]{revtex4}
%\documentclass[prb,twocolumn,showpacs,preprintnumbers,amsmath,amssymb]{revtex4}% Physical Review B

\usepackage{graphicx}% Include figure files
\usepackage{dcolumn}% Align table columns on decimal point
\usepackage{bm}% bold math
\usepackage[utf8]{inputenc}
\renewcommand{\thefootnote}{\arabic{footnote}}
%\nofiles

\begin{document}

\title{Espectros atómicos}% Force line breaks with \\

\author{Alejandro Hernández A.}%
 \email{a.hernandez105@uniandes.edu.co}
\author{Daniel Sánchez M.}%
 \email{d.sanches462@uniandes.edu.co}
\affiliation{%
Departamento de Física\\ Universidad de los Andes, Bogotá, Colombia.\\
}%


\date{13 de agosto de 2015}% It is always \today, today,
             %  but any date may be explicitly specified

\begin{abstract}
Este informe presenta los datos obtenidos al medir con un espectrómetro de prima sencillo las líneas espectrales de los siguientes gases: hidrógeno, helio, mercurio, kryptón y argón. Conociendo  la separación entre estas líneas y aplicando el modelo de Bohr, se obtuvo un valor experimental para la constante de Rydberg, así como la diferencia de energía que corresponde a cada una de ellas.
\\

%\smallskip
\noindent \textbf{Conceptos clave:} Espectrómetro, modelo de Bohr, líneas espectrales, constante de Rydberg.
\end{abstract}
                             
\maketitle

\section{\label{sec:level1}Introducción.}

Una línea espectral es una línea brillante que resalta en un espectro uniforme y continuo, que resulta de la emisión o absorción de luz en un rango de frecuencias determinado. Este tipo de líneas son típicamente usadas para identificar átomos y moléculas a partir de sus líneas espectrales características.\\

El modelo más sencillo que explica teóricamente la aparición de las líneas espectrales en el espectro atómico es el modelo de Bohr. Partiendo de la relación de de Broglie $\lambda = \frac{h}{mv}$ y de la cuantización del momento angular $L = n\frac{h}{2\pi} = n\hbar, \ n \in \mathbb{N}$, el modelo de Bohr predice que las enerías de los diversos niveles electrónicos del átomo de hidrógeno están dadas por

\begin{equation}
E_{n} = -\frac{Z^2 k^2 e^4 m_{e}}{2 \hbar^2 n^2} \approx -\frac{13.6 Z^2}{n^2}\    
\end{equation}

\noindent
donde $n$ caracteriza el nivel de energía en consideración, $Z$ es el número atómico, $k$ es la constante de Coulomb, $e$ es la carga fundamental y $m_{e}$ la masa del electrón.
\\
Ahora bien, para $Z = 1$, la energí de un fotón emitido por un átomo de hidrógeno cuando un electrón salta de un nivel $n_{i}$ a un nivel $n_{f}$ es

\begin{equation}
\Delta E = E_{i} - E_{f} = \frac{k^2 e^4 m_{e}}{2 \hbar ^2} \left(\frac{1}{n_{f}^2}- \frac{1}{n_{i}^2} \right)  
\end{equation}

Finalmente, al tener en cuenta que la energía de un fotón es $\Delta E = \frac{hc}{\lambda}$, la longitud de onda para un fotón emitido en el salto $n_{i} \rightarrow n_{f}$ es

\begin{equation}
\frac{1}{\lambda} = \frac{k^2 e^4 m_{e}}{4 \pi \hbar ^3 c} \left(\frac{1}{n_{f}^2}- \frac{1}{n_{i}^2} \right)  
\end{equation}

La anterior ecuación se conoce como la fórmula de Rydberg y 
\\

\begin{equation}
\label{rydberg}
R = \frac{k^2 e^4 m_{e}}{4 \pi \hbar ^3 c}
\end{equation}

se denomina constante de Rydberg.

\section{Montaje experimental}

Con el fin de medir las líneas espectrales de los diversos gases proporcionados, se usó un espectrómetro de prisma sencillo.\\ 

El montaje experimental y el nombre de cada elemento usado durante el laboratorio se muestra a continuación.\\

\begin{figure}[h!]
	\centering
	\includegraphics[width=0.5\textwidth]{espectrometro}
	\caption{Componentes de un espectrómetro de prima sencillo.}
\end{figure}

Los pasos seguidos durante el desarrollo del mencionado laboratorio fueron los siguientes:
\\

\begin{enumerate}
	\item \textbf{Enfoque del espectrómetro:} Antes de poner el prisma en el centro del espectrómetro y de realizar las mediciones propiamente dichas se ajustó la altura del espectrómetro para una visualización cómoda, así como los focos del colimador y del telescopio, con el fin de tener una visión lo más clara posible de las líneas espectrales.
	
	\item \textbf{Calibración del espectrómetro:} Tomando como punto de partida el tubo espectral que contenía al helio, se ubicó el prisma sobre la base giratoria y se midieron, con ayuda de una reglilla proyectada, las posiciones de las líneas del espectro de este gas. Finalmente, mediante una curva de la longitud de onda asociada a cada línea de acuerdo a su color vs la posición medida de la línea, se obtuvo una curva de calibración para el espectrómetro.
	
	\item \textbf{Obtención de la constante de Rydberg:} Tras cambiar el tuvo espectral de helio por el de hidrógeno  y usando los resultados de la curva de calibración y el modelo de Bohr, se determinó un valor experimental para la constante de Rydberg, cuyo valor teórico esta dado por \eqref{rydberg}.
	
	\item \textbf{Mediciones para lso demás gases:} Repitiendo el proceso de los dos pasos anteriores y cambiando el tuvo espectral observado, se midieron las líneas espectrales de los demás gases.
\end{enumerate}

\section{Resultados y análisis}

Los resultados de las mediciones de las múltiples líneas espectrales de los diversos gases se muestran en las siguientes tablas.\\

En lo que respecta a la calibración del espectrómetro con helio, los datos fueron los siguientes:

\begin{table}[h!]
	\caption{\label{Tabla 1}Espectro observado para el helio.}
	\begin{ruledtabular}
		\begin{tabular}{cc}
			Color de la línea&Posición(nm)\\
			\hline
			Rojo & 8.4\\
			Amarillo & 9.6\\
			Verde & 13.0\\
			Azul & 13.2\\
			Azul tenue & 14.5\\
			Morado & 16.0\\
		\end{tabular}
	\end{ruledtabular}
\end{table}
\
\\
\indent
Identificando la longitud de onda de cada una de las líneas mediante su color, se graficaron estos datos contra la posición medida de las mismas. A partir de \ref{fig 2} (OJO, NO SÉ CÓMO REFERENCIARLA)se observa una aparente relación lineal entre la longitud de onda y la posición medida de la línea, razón por la cual se hizo una regresión lineal sobre los datos. 
\\

El resultado de la regresión fue el siguiente:

\begin{equation}
\label{regresion}
\lambda = -30.228x + 895.376
\end{equation}
\noindent
donde x es la posición medida de la línea. El error estándar $s = 3.821$ y correlación $r^2 = 0.939$ justifican la relación lineal propuesta entre los datos. Dado lo anterior, la curva de calibración queda determinada por \eqref{regresion}
\begin{figure}[h!]
	\label{fig 2}
	\centering
	\includegraphics[width=0.5\textwidth]{regresion}
	\caption{Datos expermimentales para el helio.}
\end{figure}

AQUÍ HAY QUE DETERMINAR LA CONSTANTE DE RYDBERG, PERO NECESITO SU AYUDA.
\begin{table}[h!]
\caption{\label{Tabla 2}Espectro observado para el hidrógeno.}
\begin{ruledtabular}
\begin{tabular}{cc}
Color de la línea&Posición(nm)\\
\hline
Rojo & 8.4\\
Verde & 13.7\\
Morado & 17.0\\
\end{tabular}
\end{ruledtabular}
\end{table}



\begin{table}[h!]
\caption{\label{Tabla 3}Espectro observado para el mercurio.}
\begin{ruledtabular}
\begin{tabular}{cc}
Color de la línea&Posición(nm)\\
\hline
Amarillo & 10.2\\
Verde & 11.3\\
Verde tenue & 13.4\\
Morado & 16.9\\
\end{tabular}
\end{ruledtabular}
\end{table}

\begin{table}[h!]
\caption{\label{Tabla 4}Espectro observado para el argón.}
\begin{ruledtabular}
\begin{tabular}{cc}
Color de la línea&Posición(nm)\\
\hline
Rojo & 6.8\\
Naranja & 8.0\\
Verde & 9.5\\
Verde tenue & 10.9\\
Azul & 12.5\\
Morado & 16.0\\
\end{tabular}
\end{ruledtabular}
\end{table}

\begin{table}[h!]
\caption{\label{Tabla 5}Espectro observado para el kriptón.}
\begin{ruledtabular}
\begin{tabular}{cc}
Color de la línea&Posición(nm)\\
\hline
Rojo & 6.9\\
Naranja & 8.3\\
Verde & 9.2\\
Verde tenue & 10.5\\
Azul & 12.0\\
Morado & 14.9\\
\end{tabular}
\end{ruledtabular}
\end{table}

HAY QUE HABLAR UN POCO SOBRE LOS OTROS DATOS, Y MOSTRAR LAS OTRAS CONSTANTES DE RYDBERG.
HAY QUE HABLAR TAMBIÉN DE LSO PROBLEMAS ENCONTRADOS DURANTE EL EXPERIMENTO.

\section{Conclusiones}

NECESITO SU AYUDA PARA HACER ESTO

\section{Bibliografía}
\end{document}
%
% ****** End of file apssamp.tex ******
